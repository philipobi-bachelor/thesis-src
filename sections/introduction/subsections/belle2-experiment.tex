The Belle II experiment is situated at the SuperKEKB electron-positron collider in Tsukuba, Japan. 
SuperKEKB is a next-generation B-factory, a particle collider specifically designed to produce large numbers of B mesons \textemdash\ composite particles consisting of a bottom quark and a lighter quark. 
Studying these mesons is essential for testing the predictions of the Standard Model (SM) and for probing potential physics beyond it.

SuperKEKB collides electrons and positrons at center-of-mass energies tuned primarily to the $\Upsilon(4S)$ resonance, a short-lived bound state of a bottom quark and its antiquark. 
With a mass of approximately \qty{10.58}{\giga\electronvolt}, the $\Upsilon(4S)$ decays almost exclusively into particle-antiparticle pairs of neutral or charged B mesons. 
This makes it an ideal source for studying B meson properties in a clean and well-controlled experimental environment.

The central scientific objective of Belle II is to investigate the flavor sector of the Standard Model, i.e.\ interactions that distinguish between quark and lepton types (\enquote{flavors}), 
with particular emphasis on heavy flavor physics involving particles that contain bottom or charm quarks. 
These studies enable precision tests of the SM and offer sensitivity to indirect effects of new, undiscovered particles or interactions.

Of particular importance in heavy flavor physics is CP violation, an asymmetry in the behavior of particles and their antiparticles under the combined transformations of charge conjugation (C) and parity (P).
Investigations of CP violation are essential to understanding the matter-antimatter asymmetry of the universe, as this phenomenon is predicted by the SM, but not at the level observed in cosmological data.
This gap between theory and observation makes it a compelling area for discovering new physics.

To achieve its scientific goals, Belle II aims to collect an integrated luminosity\footnote{%
In particle physics, (instantaneous) luminosity $\mathcal{L}$ relates the interaction rate $dR/dt$ to the production cross-section $\sigma_p$ (measure of interaction probability): $dR/dt = \mathcal{L}\,\sigma_p$. 
While $\mathcal{L}$ determines the interaction rate, the integrated luminosity $\int\mathcal{L}\,dt$ is a measure of the total data collected, since it is proportional to the number of events observed. \cite{luminosity}
} of \qty{50}{\per\atto\barn}, representing a fifty-fold increase over the data collected by its predecessor, the Belle experiment. 
This massive dataset significantly enhances sensitivity to rare decays and subtle effects in the physics of B, D, and $\tau$ particles.

SuperKEKB operates with asymmetric beam energies ($e^-$: \qty{7}{\giga\electronvolt}, $e^+$: \qty{4}{\giga\electronvolt}), resulting in a boost of the center-of-mass system, which causes produced B mesons to travel a measurable distance in the laboratory frame before decaying. The longer decay path allows for a more accurate measurement of the decay-time difference between produced pairs of a B meson and its antiparticle, a quantity essential for studying CP asymmetries.

The combination of high luminosity, clean experimental conditions, and advanced detection capabilities establishes Belle II as a leading facility for precision measurements in flavor physics and the search for physics beyond the Standard Model.