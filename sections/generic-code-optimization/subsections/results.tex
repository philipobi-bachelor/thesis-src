From the collected evaluation and benchmarking data, I programmatically generated comprehensive results tables that present relevant information regarding GitHub Copilot's optimization performance across the different test scenarios.

Tables \ref{tab:unoptimized-test1} and \ref{tab:unoptimized-test2} present relevant result data for test files 1 and 2 respectively when compiled without optimizations (-O0 flag). 
Tables \ref{tab:optimized-test1} and \ref{tab:optimized-test2} in the appendix show the corresponding results when the code was compiled with compiler optimizations enabled (-O3 flag).

Table \ref{tab:generic-code-opt-summary} summarizes the optimization test results, showing the distribution of inefficient implementations between the two test files, and the number of examples successfully improved by each GitHub Copilot model.

\begin{table}[h]
  \centering
  \import{tables/generic-code-optimization/}{summary.tex}
  \caption{Summary of optimization improvements by model across test files}
  \label{tab:generic-code-opt-summary}
\end{table}

\paragraph{Legend for Tables 
\ref{tab:unoptimized-test1},
\ref{tab:unoptimized-test2},
\ref{tab:optimized-test1},
\ref{tab:optimized-test2}%
}
\begin{itemize}
  \item \textbf{Task number / Description}: Short description of the inefficiency in each test example; full implementation details available in the Markdown file containing all examples selected for testing% 
        \footnote{\repoRef{code-optimization-test/blob/main/examples.md}{code-optimization-test/examples.md}}

  \item \textbf{average(elapsed)}: Mean wall-clock execution time measured by nanobench for one execution of the example code

  \item \textbf{bl}: Baseline performance representing \emph{average(elapsed)} for the efficient implementation of each task

  \item \textbf{s/bl}: Performance ratio $\log_{10}(slow / bl)$, where \emph{slow} is \emph{average(elapsed)} for the inefficient implementation; a measure of performance degradation of the inefficient version relative to the efficient baseline

  \item \textbf{s?}(=slow?):  Indicator denoting which implementation version was in the test file; filled circle: inefficient version present; empty circle: efficient version present

  \item \textbf{Model columns}: Each model's results are displayed in two columns:
        \begin{itemize}
          \item \textbf{Left column}: Modification status indicator
                \begin{itemize}
                  \item Filled circle: Inefficient implementation was present and Copilot modified the code
                  \item Half-filled circle: Efficient implementation was present and Copilot modified the code
                  \item Empty circle: Copilot made no modifications to the code
                  \item Red highlighting: Inefficient implementation was present but Copilot failed to modify the code
                  \item Green highlighting: Inefficient implementation was present and Copilot modified the code
                \end{itemize}
          \item \textbf{Right column}: Performance metric
                \begin{itemize}
                  \item Value equals $\log_{10}(c / bl)$, where \emph{c} is \emph{average(elapsed)} for the example's code after Copilot's modifications
                  \item No highlighting: Efficient version was present and $\log_{10}(c / bl) \in (-0.1,\ 0.1)$
                  \item Color mapping for other cases: Green tones denote performance similar or better than baseline; yellow and red tones show performance degradation
                \end{itemize}
                \item \textbf{Model abbreviations}: c-s-4 = Claude Sonnet 4, g-2.5-p = Gemini 2.5 Pro
        \end{itemize}
  \item \textbf{Note}: I implemented cell coloring to enhance the interpretability of model performance across the large dataset by highlighting patterns that may be difficult to detect from the raw data alone.
\end{itemize}

\begin{table}
  \centering
  \import{tables/generic-code-optimization/}{unoptimized-test1.tex}
  \caption{Copilot optimization results for Test File 1 (benchmarks compiled with -O0)}
  \label{tab:unoptimized-test1}
\end{table}

\begin{table}
  \centering
  \import{tables/generic-code-optimization/}{unoptimized-test2.tex}
  \caption{Copilot optimization results for Test File 2 (benchmarks compiled with -O0)}
  \label{tab:unoptimized-test2}
\end{table}

\subsubsection*{Notable Observations}

The evaluation results demonstrate clear performance differences between the underlying models powering GitHub Copilot.

Claude Sonnet 4 emerged as the strongest performer, successfully detecting and optimizing the highest number of inefficient examples across both test files (Tab.\ \ref{tab:generic-code-opt-summary}).

This superior performance is further corroborated by the detailed benchmarking data, which shows that code implementations modified by the Claude Sonnet-powered Copilot typically achieved at least baseline performance levels, with multiple cases even demonstrating improvements beyond the baseline efficient versions (Tab.\ \ref{tab:unoptimized-test1}, \ref{tab:unoptimized-test2}).

\vspace{\baselineskip}
Example 43 in test file 1 illustrates a case where Claude Sonnet 4 successfully identified an optimization opportunity but did not achieve sufficient improvement to reach baseline performance (Tab.\ \ref{tab:unoptimized-test1}).
The inefficient version employs a std::map for storing and retrieving values, using dense integer keys ranging from 0 to 9999.
The efficient reference implementation replaces this structure with a std::vector, utilizing the integer key directly as a vector index for value access.

Claude Sonnet 4's optimization involved replacing the std::map with a std::unordered\_map to enable faster key access, while additionally performing accumulation of the stored values directly during insertion, instead of looping over the keys twice.
Although benchmarking data confirms this modification improves performance, the implementation still exhibits considerable performance overhead compared to the vector-based approach.

\vspace{\baselineskip}
Example 34 demonstrates a more sophisticated optimization where Claude Sonnet 4 effectively \enquote{outsmarted} the test design.
This example involves repeated string lookups in a set, with the inefficient reference implementation using std::set and the efficient version employing std::unordered\_set for constant-time lookup complexity.

In both test scenarios, with either the inefficient version or the efficient reference version present, Claude Sonnet 4 recognized that the search target remained constant across all iterations.
Rather than optimizing the data structure, it restructured the code to perform the search operation only once and multiply the result by the number of iterations.
This approach yielded dramatic performance improvements, reducing runtime by more than four orders of magnitude compared to the efficient baseline.

Similar \enquote{outsmarting} behavior occurred in most cases where significant runtime improvements beyond the baseline were achieved, not exclusively with Claude Sonnet 4 but also with other models, particularly o4-mini and Gemini 2.5 Pro.
For instance, in example 34, o4-mini implemented the identical optimization as Claude Sonnet 4 when the inefficient version was present, though notably failed to achieve further improvements when the reference efficient version was included in the test file (cf.\ Lst.\ \ref{lst:outsmarting-o4-mini}, Tab.\ \ref{tab:unoptimized-test1}, \ref{tab:unoptimized-test2}).

\vspace{\baselineskip}
One exception in the evaluation data concerns example 50 in test file 2 for Claude Sonnet 4.
While the model successfully modified and optimized the inefficient example, it did so by significantly altering the code's functionality (Lst.\ \ref{lst:sonnet-fibonacci}).
Consequently, I classified this as \enquote{not fixing the code} despite benchmarking results showing performance gains from the optimization (Tab.\ \ref{tab:unoptimized-test2}).

\vspace{\baselineskip}
The results suggest that more common inefficiencies are easier for models to detect and improve.
For example, the performance issue introduced by column-major matrix traversal (Example 44, Tab.\ \ref{tab:unoptimized-test1}) was successfully identified and optimized by all tested models, even GPT-4o.

\vspace{\baselineskip}
Based on the count of detected and improved inefficient examples, as well as a qualitative review of the comprehensive results tables, Gemini 2.5 Pro ranks second in performance, o4-mini third, and GPT-4o last by a wide margin, having detected only 9 out of 50 inefficient examples.

\vspace{\baselineskip}
Comparing benchmarking data between -O0 and -O3 compilation flags reveals that while absolute runtimes differ substantially, the performance ratios between inefficient and efficient versions do not show any drastic changes: 
Many slow examples maintain logarithmic ratios near zero compared to the fast baseline across both compilation settings.

Several factors may explain these observations:
\begin{itemize}[topsep=0pt]
  \item Compilation with -O0 disables some but not all optimizations
  \item The primary performance overhead may stem from benchmarking instrumentation or print statements, making the effect of the intentionally introduced inefficiencies insignificant in comparison
  \item The overall quality of examples is limited, as they were not hand-crafted to maximize performance differences between inefficient and efficient versions
\end{itemize}

\vspace{\baselineskip}
Despite these limitations, I consider the data collected in this evaluation to provide a meaningful assessment of the selected AI models' code optimization capabilities.

\begin{lstbox}{%
    \captionof{customlst}{
      \enquote{Outsmarting} behavior displayed by o4-mini\footnotemark, (analogous for Claude Sonnet 4)
    }
    \label{lst:outsmarting-o4-mini}
  }
  \import{listings/v0fitter/agent-edits/}{outsmarting-o4-mini.tex}
\end{lstbox}%
\footnotetext{\repoRef{code-optimization-test/blob/main/chat-logs/test1/o4-mini.md}{code-optimization-test/chat-logs/test1/o4-mini.md}}

\begin{lstbox}{%
    \captionof{customlst}{
      Claude Sonnet 4 modification restricts function to $n \leq 40$ and indiscriminately computes the first 41 Fibonacci numbers\footnotemark
    }
    \label{lst:sonnet-fibonacci}
  }
  \import{listings/v0fitter/agent-edits/}{sonnet-fibonacci.tex}
\end{lstbox}%
\footnotetext{\repoRef{code-optimization-test/blob/main/chat-logs/test2/claude-sonnet-4.md}{code-optimization-test/chat-logs/test2/claude-sonnet-4.md}}